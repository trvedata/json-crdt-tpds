\documentclass[a4paper,twocolumn,10pt]{article}
\usepackage[utf8]{inputenc}
\usepackage{amsmath} % align environment
\usepackage{amssymb} % mathbb
\usepackage{mathptmx} % times roman, including math
\usepackage{bussproofs} % notation for inference rules
\usepackage[hyphens]{url}
\usepackage{doi}
\usepackage{hyperref}
\usepackage[numbers,sort]{natbib}
\hyphenation{da-ta-cen-ter da-ta-cen-ters time-stamp}
\frenchspacing

\newif\ifproofdraft
\proofdraftfalse

\newcommand{\evalto}{\;\Longrightarrow\;}

% Placeholder character like \textvisiblespace, but works in math mode
\newcommand{\placeholder}{%
  \makebox[0.7em]{%
    \kern.07em
    \vrule height.3ex
    \hrulefill
    \vrule height.3ex
    \kern.07em
  }%
}

\newcommand{\multialign}[2]{%
  \multispan{#1}\mbox{$\displaystyle{}#2$}%
}

\begin{document}
\sloppy
\title{A Conflict-Free Replicated JSON Datatype}
\author{}
\maketitle

\subsection*{Abstract}

\section{Introduction}

CRDTs~\cite{Shapiro:2011wy,Roh:2011dw}

% TODO papers on formal verification of CRDTs or similar:
% Gotsman et al, "cause I'm strong enough", POPL 2016
% various papers by Sebastian Burckhardt
% Roh et al's tech report (lemmas 8-13, theorem 4)
% Zeller & Bieniusa
%
% Look at KC's Quelea (also in DEBull March issue)

\section{Operational Semantics}

We go about defining the semantics for collaboratively editable data structures as follows. Firstly, we define a simple command language that is executed locally at any of the peers, and which allows the local state of that peer's copy of the document to be queried and modified. Performing read-only queries has no side-effects, but modifying the document has the effect of producing \emph{operations} describing the mutation. Those operations are applied to the local copy of the document, and also enqueued for broadcasting to other peers.

\subsection{Data model}

Our data model for a document is based on JSON. It consists of a tree with the following types of node:

\begin{description}
\item[Map:] A branch node whose children are not ordered, and where each child is labelled with a string \emph{key}. A key uniquely identifies one of the children. Keys are immutable, but values are mutable, and key-value mappings can be added and removed from the map. A JSON map is also known as an \emph{object}.
\item[List:] A branch node whose children have an order defined by the application. The list can be mutated by inserting or deleting list elements. A JSON list is also known as an \emph{array}.
\item[Register:] A leaf node containing a primitive value (string, number, boolean or null). Although primitives are themselves immutable, a register can be mutated by assigning a new value to it.
\end{description}

Thus, in the tree of a JSON document, all the interior branch nodes are either maps or lists, and all the leaf nodes are registers. Maps and lists may be nested within each other arbitrarily. For simplicity we do not define a schema language or static type-checking rules, but such restrictions of the document structure could be added easily. Our approach also easily generalizes to other replicated datatypes such as counters or sets, which we do not spell out in detail in the interest of brevity.

\begin{figure}
\centering
\begin{tabular}{rcll}
CMD & ::= & \textsf{let} $x$ = EXPR & $x \in \mathrm{VAR}$ \\
& $|$ & $\mathrm{EXPR} := v$ & $v \in \mathrm{VAL}$ \\
& $|$ & $\mathrm{EXPR}.\mathsf{insert}(v)$ & $v \in \mathrm{VAL}$ \\
& $|$ & $\mathrm{EXPR}.\mathsf{delete}$ \\
& $|$ & \textsf{yield} \\
& $|$ & CMD; CMD \vspace{0.5em}\\
EXPR & ::= & \textsf{doc} \\
& $|$ & $x$ & $x \in \mathrm{VAR}$ \\
& $|$ & EXPR.\textsf{get}(\textit{key}) & $\mathit{key} \in \mathrm{String}$ \\
& $|$ & EXPR.\textsf{iter} \\
& $|$ & EXPR.\textsf{next} \\
& $|$ & EXPR.\textsf{values} \vspace{0.5em}\\
VAR & ::= & ${x_1, x_2, \dots}$ \vspace{0.5em}\\
VAL & ::= & $n$ & $n \in \mathrm{Number}$ \\
& $|$ & \verb|"str"| & $\mathtt{str} \in \mathrm{String}$ \\
& $|$ & \verb|true| $|$ \verb|false| $|$ \verb|null| \\
& $|$ & \verb|{}| $|$ \verb|[]|
\end{tabular}
\caption{Syntax of command language for querying and modifying a document}\label{fig:local-syntax}
\end{figure}

\begin{figure}
\centering
\begin{verbatim}
doc := {};
let list = doc.get("shopping").iter;
list.insert("eggs");
let eggs = list.next;
eggs.insert("milk");
list.insert("cheese");

// Final state:
{"shopping": ["cheese", "eggs", "milk"]}

eggs.values // evaluates to {"eggs"}
eggs.next.values // evaluates to {"milk"}
\end{verbatim}
\caption{Example of programmatically constructing a JSON document}\label{fig:make-doc}
\end{figure}

\subsection{Command language}

The syntax of the command language is given in Figure~\ref{fig:local-syntax}. It is not a full programming language, but rather an API through which the document state is queried and modified. We assume that the program accepts user input and issues a (possibly infinite) sequence of commands to the API. We model only the semantics of those commands, and do not assume anything about the program in which the command language is embedded.

We first explain the language informally, before giving its formal semantics. The expression construct EXPR is used to query the state of a document. An expression starts with either the special token \textsf{doc}, identifying the root of the JSON document tree, or a variable $x$ that was previously defined in a \textsf{let} command. The expression then continues with a sequence of method calls: $\mathsf{get}(\mathit{key})$ selects a key within a map, \textsf{iter} starts iterating over an ordered list, \textsf{next} moves to the next element of an ordered list, and \textsf{values} returns the value of the register at the current position within the document. (\textsf{values} is not defined for map or list nodes.)

A command CMD either sets the value of a local variable (\textsf{let}), performs network communication (\textsf{yield}), or modifies the document. A document can be modified by assigning the value of a register (using the assignment operator :=), by inserting an element into a list (\textsf{insert}), or by deleting an element from a list or a map (\textsf{delete}). The preceding expression EXPR acts as a cursor, identifying the part of the document being modified.

Figure~\ref{fig:make-doc} shows an example sequence of commands that constructs a new document representing a shopping list. First \textsf{doc} is set to \verb|{}|, the empty map literal. The second line navigates to the key \verb|"shopping"| and calls \textsf{iter}, which treats the value at that key as a list and selects the head of the list. If the key does not exist, it is implicitly created and set to the empty list. Finally, three items are inserted into the list. The \textsf{insert} command adds a new list element \emph{after} the current cursor position, or at the head if the cursor is at the head of the list. The variable \verb|list| refers to the head, so cheese is inserted before eggs, but the variable \verb|eggs| refers to the list element ``eggs'', so milk is inserted after eggs.

A few features of this language deliberately differ from most mainstream programming languages: keys in maps are implicitly created when they are first accessed, so there is no need for a command to put a new key-value pair into a map; lists can only be navigated by iteration (\textsf{next}) but not by index; and the language has literals for creating empty maps and lists, but not for non-empty collections. As we shall see later, these features are helpful for achieving desirable semantics in the presence of concurrent modifications.

\subsubsection{Semantics of expression evaluation}

\begin{figure}
\centering \begin{alignat*}{3}
& \multialign{5}{A_p = \{\; \mathsf{mapT}(\mathsf{doc}) \mapsto \{\;
    \mathsf{listT}(\text{``shopping''}) \mapsto \{ } \\
&&&& \mathsf{next}(\mathsf{head}) & \mapsto \mathit{id}_3, \\
&&&& \mathsf{regT}(\mathit{id}_3) & \mapsto \{\,\mathit{id}_3 \mapsto \text{``cheese''})\,\}, \\
&&&& \mathsf{next}(\mathit{id}_3) & \mapsto \mathit{id_1}, \\
&&&& \mathsf{regT}(\mathit{id}_1) & \mapsto \{\,\mathit{id}_1 \mapsto \text{``eggs''})\,\}, \\
&&&& \mathsf{next}(\mathit{id}_1) & \mapsto \mathit{id_2}, \\
&&&& \mathsf{regT}(\mathit{id}_2) & \mapsto \{\,\mathit{id}_2 \mapsto \text{``milk''})\,\}, \\
&&&& \mathsf{next}(\mathit{id}_2) & \mapsto \mathsf{tail} \\
&&& \}\;\}, \\
&& \mathsf{list} & \multialign{3}{\mapsto \mathsf{cursor}(\langle \mathsf{mapT}(\mathsf{doc}),%
    \mathsf{listT}(\text{``shopping''})\rangle,\, \mathsf{head}),} \\
&& \mathsf{eggs} & \multialign{3}{\mapsto \mathsf{cursor}(\langle \mathsf{mapT}(\mathsf{doc}),%
    \mathsf{listT}(\text{``shopping''})\rangle,\, \mathit{id}_1)} \\
& \}
\end{alignat*}
\caption{Internal state $A_p$ of peer $p$ after the execution of the commands in Figure~\ref{fig:make-doc}.}\label{fig:state-example}
\end{figure}

\begin{figure*}
\begin{center}
\AxiomC{$\mathit{cmd}_1 \mathbin{:} \mathrm{CMD}$}
\AxiomC{$A_p,\, \mathit{cmd}_1 \evalto A_p'$}
\LeftLabel{\textsc{Exec}}
\BinaryInfC{$A_p,\, \langle \mathit{cmd}_1 \mathbin{;} \mathit{cmd}_2 \mathbin{;} \dots \rangle
    \evalto A_p',\, \langle \mathit{cmd}_2 \mathbin{;} \dots \rangle$}
\DisplayProof\hspace{4em}
%
\AxiomC{}
\LeftLabel{\textsc{Doc}}
\UnaryInfC{$A_p,\, \mathsf{doc} \evalto \mathsf{cursor}(\langle\rangle,\, \mathsf{doc})$}
\DisplayProof\proofSkipAmount
\end{center}

\begin{center}
\AxiomC{$A_p,\, \mathit{expr} \evalto \mathit{cur}$}
\LeftLabel{\textsc{Let}}
\UnaryInfC{$A_p,\, \mathsf{let}\; x = \mathit{expr} \evalto A_p[\,x \,\mapsto\, \mathit{cur}\,]$}
\DisplayProof\hspace{3em}
%
\AxiomC{$x \in \mathrm{dom}(A_p)$}
\LeftLabel{\textsc{Var}}
\UnaryInfC{$A_p,\, x \evalto A_p(x)$}
\DisplayProof\proofSkipAmount
\end{center}

\begin{prooftree}
\AxiomC{$A_p,\, \mathit{expr} \evalto \mathsf{cursor}(\langle k_1, \dots, k_{n-1} \rangle,\, k_n)$}
\LeftLabel{\textsc{Get}}
\UnaryInfC{$A_p,\, \mathit{expr}.\mathsf{get}(\mathit{key}) \evalto
    \mathsf{cursor}(\langle k_1, \dots, k_{n-1}, \mathsf{mapT}(k_n) \rangle,\, \mathit{key})$}
\end{prooftree}

\begin{prooftree}
\AxiomC{$A_p,\, \mathit{expr} \evalto \mathsf{cursor}(\langle k_1, \dots, k_{n-1} \rangle,\, k_n)$}
\LeftLabel{\textsc{Iter}}
\UnaryInfC{$A_p,\, \mathit{expr}.\mathsf{iter} \evalto
    \mathsf{cursor}(\langle k_1, \dots, k_{n-1}, \mathsf{listT}(k_n) \rangle,\, \mathsf{head})$}
\end{prooftree}

\begin{center}
\AxiomC{$A_p,\, \mathit{expr} \evalto \mathit{cur}$}
\AxiomC{$A_p,\, \mathit{cur}.\mathsf{next} \evalto \mathit{cur}'$}
\LeftLabel{$\textsc{Next}_1$}
\BinaryInfC{$A_p,\, \mathit{expr}.\mathsf{next} \evalto \mathit{cur}'$}
\DisplayProof\hfill
%
\AxiomC{$\mathsf{next}(k) \in \mathrm{dom}(\mathit{ctx})$}
\AxiomC{$\mathit{ctx}(\mathsf{next}(k)) = k'$}
\LeftLabel{$\textsc{Next}_2$}
\BinaryInfC{$\mathit{ctx},\, \mathsf{cursor}(\langle\rangle,\, k).\mathsf{next} \evalto
    \mathsf{cursor}(\langle\rangle,\, k')$}
\DisplayProof\proofSkipAmount
\end{center}

\begin{prooftree}
\AxiomC{$k_1 \in \mathrm{dom}(\mathit{ctx})$}
\AxiomC{$\mathit{ctx}(k_1),\, \mathsf{cursor}(\langle k_2, \dots, k_{n-1} \rangle,\, k_n).\mathsf{next}
    \evalto \mathsf{cursor}(\langle k_2, \dots, k_{n-1} \rangle,\, k_n')$}
\LeftLabel{$\textsc{Next}_3$}
\BinaryInfC{$\mathit{ctx},\, \mathsf{cursor}(\langle k_1, k_2, \dots, k_{n-1} \rangle,\, k_n).\mathsf{next}
    \evalto \mathsf{cursor}(\langle k_1, k_2, \dots, k_{n-1} \rangle,\, k_n')$}
\end{prooftree}

\begin{prooftree}
\AxiomC{$A_p,\, \mathit{expr} \evalto \mathit{cur}$}
\AxiomC{$A_p,\, \mathit{cur}.\mathsf{values} \evalto \mathit{val}$}
\LeftLabel{$\textsc{Val}_1$}
\BinaryInfC{$A_p,\, \mathit{expr}.\mathsf{values} \evalto \mathit{val}$}
\end{prooftree}

\begin{prooftree}
\AxiomC{$\mathsf{regT}(k) \in \mathrm{dom}(\mathit{ctx})$}
\AxiomC{$\mathit{val} = \mathrm{range}(\mathit{ctx}(\mathsf{regT}(k)))$}
\LeftLabel{$\textsc{Val}_2$}
\BinaryInfC{$\mathit{ctx},\, \mathsf{cursor}(\langle\rangle,\, k).\mathsf{values} \evalto \mathit{val}$}
\end{prooftree}

\begin{prooftree}
\AxiomC{$k_1 \in \mathrm{dom}(\mathit{ctx})$}
\AxiomC{$\mathit{ctx}(k_1),\, \mathsf{cursor}(\langle k_2, \dots, k_{n-1} \rangle,\, k_n).\mathsf{values}
    \evalto \mathit{val}$}
\LeftLabel{$\textsc{Val}_3$}
\BinaryInfC{$\mathit{ctx},\, \mathsf{cursor}(\langle k_1, k_2, \dots, k_{n-1} \rangle,\, k_n).\mathsf{values}
    \evalto \mathit{val}$}
\end{prooftree}
\caption{Rules for evaluating an expression to obtain a cursor}\label{fig:expr-rules}
\end{figure*}

The state of peer $p$ is described by $A_p$, a finite partial function. The semantics of the command language are defined by rules that inspect and modify this local state $A_p$, and which are independent of the state $A_{p'}$ of any other peer $p'$. The only communication between peers occurs in the evaluation of the \textsf{yield} command, which we discuss later. For now, we concentrate on the execution of commands at a single peer $p$.

An illustrative example of the peer state $A_p$ is given in Figure~\ref{fig:state-example}, which corresponds to the shopping list example of Figure~\ref{fig:make-doc}. For each local variable defined with a \textsf{let} command, $A_p$ maps the variable name to a \emph{cursor}, which identifies a position in the document as described below. In addition, $A_p$ maps the single atom \textsf{mapT(doc)} to a nested partial function representing the contents of the document. \textsf{mapT} denotes that the document \textsf{doc} is of type map. The only map entry is the key ``shopping'' of type \textsf{listT}. The list is represented in a manner resembling a linked list, with each list element assigned a unique identifier ($\mathit{id}_1, \mathit{id}_2, \mathit{id}_3$), and special \textsf{head} and \textsf{tail} atoms denoting the beginning and end of the list, respectively.

Figure~\ref{fig:expr-rules} gives the rules for evaluating EXPR expressions in the command language, which are evaluated in the context of the local peer state $A_p$. The \textsc{Exec} rule formalizes the assumption that commands are executed sequentially. The \textsc{Let} rule allows the program to define a local variable, which is added to the local state, and the corresponding \textsc{Var} rule allows the program to retrieve the value of a previously defined variable.

The rules in Figure~\ref{fig:expr-rules} show how an expression is evaluated to a \emph{cursor} atom. A cursor unambiguously identifies a particular position in a JSON document by describing a path from the root of the document tree to some branch or leaf node. A cursor consists only of immutable keys and identifiers, so it can be sent over the network to another peer, where it can be used to locate the same position in the document.

For example,
\[ \mathsf{cursor}(\langle \mathsf{mapT}(\mathsf{doc}), \mathsf{listT}(\text{``shopping''}) \rangle,\, \mathit{id}_1) \]
is a cursor representing the list element \verb|"eggs"| in Figure~\ref{fig:make-doc}. It can be interpreted as a path through the structure in Figure~\ref{fig:state-example}, read from left to right: starting from the \textsf{doc} map at the root, it traverses through the map entry ``shopping'' of type \textsf{listT}, and finishes with the list element with identifier $\mathit{id}_1$.

In general, $\mathsf{cursor}(\langle k_1, \dots, k_{n-1} \rangle,\, k_n)$ consists of a (possibly empty) vector of keys $\langle k_1, \dots, k_{n-1} \rangle$, and a final key $k_n$ (which is always present). $k_n$ can be thought of as the final element of the vector, with the distinction that it is not tagged with a datatype, whereas the elements of the vector are tagged with the datatype of the branch node, either \textsf{mapT} or \textsf{listT}.

The \textsc{Doc} rule in Figure~\ref{fig:expr-rules} defines the simplest cursor $\mathsf{cursor}(\langle\rangle,\, \mathsf{doc})$, referencing the root of the document using the special atom \textsf{doc}. The \textsc{Get} rule navigates a cursor to a particular key within a map. For example, the expression \verb|doc.get("shopping")| evaluates to $\mathsf{cursor}(\langle \mathsf{mapT}(\mathsf{doc}) \rangle,\, \text{``shopping''})$ by applying the \textsc{Doc} and \textsc{Get} rules. Note that the expression \verb|doc.get| implicitly asserts that \textsf{doc} is of type \textsf{mapT}, and this assertion is encoded in the cursor.

The \textsc{Iter} rule shifts the cursor into a list and positions it at the \textsf{head} of the list. This rule applies even if the list is empty or nonexistent in $A_p$. The three rules $\textsc{Next}_{1,2,3}$ handle iteration through a linked list by setting the final key in the cursor to the identifier of the next list element. The \textsc{Next} rules apply only if the list exists in $A_p$, and the $\textsc{Next}_3$ recursively descends the local state according to the vector of keys in the cursor.

Finally, the $\textsc{Val}_{1,2,3}$ rules allow a program to read the contents of a register at a particular cursor position, using a similar recursive rule structure as the \textsc{Next} rules. A register is expressed using the \textsf{regT} type annotation in the local state, and the \textsc{Val} rules only apply if the register identified by the cursor exists in $A_p$. Although a peer can only assign a single value to a register, a register can nevertheless contain multiple values if multiple peers concurrently assign values to it -- an issue we will explore in greater depth later.

\subsection{Generating operations}

When commands mutate the state of the document, they generate \emph{operations} that describe the mutation. In our semantics, a command never directly modifies the local peer state $A_p$, but only generates an operation. That operation is then immediately applied to $A_p$ so that it takes effect locally, and the operation is also asynchronously broadcast to the other peers. A peer applies operations received from remote peers when its causal dependencies are satisfied, as detailed below.

\subsubsection{Lamport timestamps}

Every operation in our model is given a unique identifier, which is used in the local state and in cursors. For example, in Figure~\ref{fig:state-example}, $\mathit{id}_{1,2,3}$ are used to identify list elements and also the values of registers. Those identifiers $\mathit{id}_{1,2,3}$ are in fact the identifiers of the operations that inserted the list elements.

In order to generate globally unique operation identifiers without requiring synchronous coordination between peers we use Lamport timestamps~\cite{Lamport:1978jq}. A Lamport timestamp is a pair $(c, p)$ where $p$ is the unique identifier of the peer on which the edit is made (for example, a cryptographic hash of its public key), and $c$ is a counter that is stored at each peer and incremented for every operation. Since each peer generates a strictly monotonically increasing sequence of counter values $c$, the pair $(c, p)$ is unique.

If a peer receives an operation with a counter value $c$ that is greater than the locally stored counter value, the local counter is increased to the value of the incoming counter. This ensures that if operation $o_1$ happened before $o_2$ (that is, the peer that generated $o_2$ had received and processed $o_1$ before $o_2$ was generated), then $o_2$ must have a greater counter value than $o_1$. Only concurrent operations can have equal counter values.

We can thus define a total ordering $<$ for Lamport timestamps:
\[ (c_1, p_1) < (c_2, p_2) \;\text{ iff }\; (c_1 < c_2) \vee (c_1 = c_2 \wedge p_1 < p_2). \]
If one operation happened before another, this ordering is consistent with causality (the earlier operation has a lower timesetamp). If two operations are concurrent, their order according to $<$ is arbitrary but deterministic. This ordering property is important for our definition of the semantics of ordered lists.

\subsubsection{Operation structure}

An operation is a tuple of the form
\begin{alignat*}{2}
& \mathsf{op}( \\
&& \mathit{id} &: \mathbb{N} \times \mathrm{PeerID}, \\
&& \mathit{deps} &: \mathcal{P}(\mathbb{N} \times \mathrm{PeerID}), \\
&& \mathit{cur} &: \mathsf{cursor}(\langle k_1, \dots, k_{n-1} \rangle,\, k_n), \\
&& \mathit{mut} &: \mathsf{insert}(v) \mid \mathsf{delete} \mid \mathsf{assign}(v) \\
& )
\end{alignat*}
where $\mathit{id}$ is the Lamport timestamp that uniquely identifies the operation, $\mathit{cur}$ is the cursor describing the position in the document being modified, and $\mathit{mut}$ is the mutation that was requested at the specified position.

$\mathit{deps}$ is the set of causal dependencies of the operation, given as a set of Lamport timestamps. The semantics below defines $\mathit{deps}$ to be the set of all operation IDs that had been applied to the document at the time when the operation was generated. In a real implementation, this set would become impracticably large, so a compact representation of causal history would be used instead -- for example, version vectors or dotted version vectors (TODO citations). However, to avoid ambiguity in our semantics we give the dependencies as a simple set of operation IDs.

The purpose of the causal dependencies $\mathit{deps}$ is to impose a partial ordering on operations: an operation can only be applied after all operations that ``happened before'' it have been applied. In particular, this means that the sequence of operations generated at one particular peer will be applied in the same order at every other peer. Operations that are concurrent, i.e. where there is no causal dependency, can be applied in any order.

\subsubsection{Semantics of generating operations}

\begin{figure*}
\centering
\begin{prooftree}
\AxiomC{$A_p,\, \mathit{expr} \evalto \mathit{cur}$}
\AxiomC{$\mathit{val}: \mathrm{VAL} \,\wedge\, \mathit{val} \not= \texttt{[]}
    \,\wedge\, \mathit{val} \not= \texttt{\string{\string}} $}
\AxiomC{$A_p,\, \mathsf{makeOp}(\mathit{cur}, \mathsf{assign}(\mathit{val})) \evalto A_p'$}
\LeftLabel{\textsc{Assign}}
\TrinaryInfC{$A_p,\, \mathit{expr} \,\text{ := }\, \mathit{val} \evalto A_p'$}
\end{prooftree}

\begin{prooftree}
\AxiomC{$A_p,\, \mathit{expr} \evalto \mathit{cur}$}
\AxiomC{$\mathit{val}: \mathrm{VAL}$}
\AxiomC{$A_p,\, \mathsf{makeOp}(\mathit{cur}, \mathsf{insert}(\mathit{val})) \evalto A_p'$}
\LeftLabel{\textsc{Insert}}
\TrinaryInfC{$A_p,\, \mathit{expr}.\mathsf{insert}(\mathit{val}) \evalto A_p'$}
\end{prooftree}

\begin{prooftree}
\AxiomC{$A_p,\, \mathit{expr} \evalto \mathit{cur}$}
\AxiomC{$A_p,\, \mathsf{makeOp}(\mathit{cur}, \mathsf{delete}) \evalto A_p'$}
\LeftLabel{\textsc{Delete}}
\BinaryInfC{$A_p,\, \mathit{expr}.\mathsf{delete} \evalto A_p'$}
\end{prooftree}

\begin{prooftree}
\AxiomC{$\mathit{ctr} = \mathrm{max}(\{0\} \,\cup\, \{ c_i \mid (c_i, p_i) \in A_p(\mathsf{ops}) \}$}
\AxiomC{$A_p,\, \mathsf{apply}(\mathsf{op}((\mathit{ctr} + 1, p), A_p(\mathsf{ops}),
    \mathit{cur}, \mathit{mut})) \evalto A_p'$}
\LeftLabel{\textsc{Make-Op}}
\BinaryInfC{$A_p,\, \mathsf{makeOp}(\mathit{cur}, \mathit{mut}) \evalto A_p'$}
\end{prooftree}

\begin{prooftree}
\AxiomC{$A_p,\, \mathit{op} \evalto A_p'$}
\LeftLabel{\textsc{Apply-Local}}
\UnaryInfC{$A_p,\, \mathsf{apply}(\mathit{op}) \evalto A_p'[\,
    \mathsf{queue} \,\mapsto\, A_p'(\mathsf{queue}) \,\cup\, \{\mathit{op}\},\;
    \mathsf{ops} \,\mapsto\, A_p'(\mathsf{ops}) \,\cup\, \{\mathit{op.id}\}\,]$}
\end{prooftree}

\begin{prooftree}
\AxiomC{$\mathit{op} \in A_p(\mathsf{recv})$}
\AxiomC{$\mathit{op.id} \notin A_p(\mathsf{ops})$}
\AxiomC{$\mathit{op.deps} \subseteq A_p(\mathsf{ops})$}
\AxiomC{$A_p,\, \mathit{op} \evalto A_p'$}
\LeftLabel{\textsc{Apply-Remote}}
\QuaternaryInfC{$A_p,\, \mathsf{yield} \evalto
    A_p'[\,\mathsf{ops} \,\mapsto\, A_p'(\mathsf{ops}) \,\cup\, \{\mathit{op.id}\}\,]$}
\end{prooftree}

\begin{prooftree}
\AxiomC{}
\LeftLabel{\textsc{Send}}
\UnaryInfC{$A_p,\, \mathsf{yield} \evalto
    A_p[\,\mathsf{send} \,\mapsto\, A_p(\mathsf{send}) \,\cup\, A_p(\mathsf{queue})\,]$}
\end{prooftree}

\begin{prooftree}
\AxiomC{$q: \mathrm{PeerID}$}
\LeftLabel{\textsc{Recv}}
\UnaryInfC{$A_p,\, \mathsf{yield} \evalto
    A_p[\,\mathsf{recv} \,\mapsto\, A_p(\mathsf{recv}) \,\cup\, A_q(\mathsf{send})\,]$}
\end{prooftree}

\begin{prooftree}
\AxiomC{$A_p,\, \mathsf{yield} \evalto A_p'$}
\AxiomC{$A_p',\, \mathsf{yield} \evalto A_p''$}
\LeftLabel{\textsc{Yield}}
\BinaryInfC{$A_p,\, \mathsf{yield} \evalto A_p''$}
\end{prooftree}
\caption{Rules for generating, sending, and receiving operations}
\label{fig:send-recv}
\end{figure*}

The formal evaluation rules for commands are given in Figure~\ref{fig:send-recv}. The \textsc{Assign}, \textsc{Insert} and \textsc{Delete} rules define how these respective commands mutate the document: all three delegate to the \textsc{Make-Op} rule to generate and apply the operation. \textsc{Make-Op} generates a new Lamport timestamp by choosing a counter value that is 1 greater than any existing counter in $A_p(\mathsf{ops})$, the set of all operation IDs that have been applied to the document.

\textsc{Make-Op} constructs an \textsf{op()} tuple of the form described above, and delegates to the \textsc{Apply-Local} rule to process the operation. \textsc{Apply-Local} does three things: it evaluates the operation to produce a modified local state $A_p'$, it adds the operation to the queue of generated operations $A_p(\mathsf{queue})$, and it adds the operation ID to the set of processed operations $A_p(\mathsf{ops})$.

The remaining rules in Figure~\ref{fig:send-recv}, \textsc{Apply-Remote}, \textsc{Send}, \textsc{Recv} and \textsc{Yield} define the semantics of the \textsf{yield} command. Since any of these rules can be used to evaluate \textsf{yield}, the semantics is nondeterministic, which models the asynchronicity of the network between peers: a message sent by one peer arrives at another peer at some arbitrarily later point in time, and there is no mesage ordering guarantee in the network.

The \textsc{Send} rule takes any operations that were placed in $A_p(\mathsf{queue})$ by \textsc{Apply-Local} and adds them to a send buffer $A_p(\mathsf{send})$. Correspondingly, the \textsc{Recv} rule takes operations in the send buffer of peer $q$ and adds them to the receive buffer $A_p(\mathsf{recv})$ of peer $p$. This is the only rule that involves more than one peer, and it models all network communication.

Once an operation appears in the receive buffer $A_p(\mathsf{recv})$, the rule \textsc{Apply-Remote} may apply. Under the preconditions that the operation has not already been processed and that its causal dependencies are satisified, \textsc{Apply-Remote} evaluates the operation in the same way as \textsc{Apply-Local}, and adds the operation ID to the set of processed operations $A_p(\mathsf{ops})$.

The actual document modifications are performed by evaluating the operations, which we discuss in the next section.

\subsection{Applying operations}

% TODO

\begin{figure*}
\begin{prooftree}
\AxiomC{$k_1 \in \mathrm{dom}(\mathit{ctx})$}
\AxiomC{$\mathit{ctx}(k_1),\, \mathsf{op}(\mathit{id}, \mathit{deps},
    \mathsf{cursor}(\langle k_2, \dots, k_{n-1} \rangle,\, k_n), \mathit{mut}) \evalto \mathit{state}$}
\LeftLabel{\textsc{Descend}}
\BinaryInfC{$\mathit{ctx},\, \mathsf{op}(\mathit{id}, \mathit{deps},
    \mathsf{cursor}(\langle k_1, k_2, \dots, k_{n-1} \rangle,\, k_n), \mathit{mut}) \evalto
    \mathit{ctx}[\, k_1 \,\mapsto\, \mathit{state} \,]$}
\end{prooftree}

\begin{prooftree}
\AxiomC{$k_1 \notin \mathrm{dom}(\mathit{ctx})$}
\AxiomC{$k_1 = \mathsf{mapT}(k_\mathit{new})$}
\AxiomC{$\{\},\, \mathsf{op}(\mathit{id}, \mathit{deps},
    \mathsf{cursor}(\langle k_2, \dots, k_{n-1} \rangle,\, k_n), \mathit{mut}) \evalto \mathit{state}$}
\LeftLabel{\textsc{Def-Map}}
\TrinaryInfC{$\mathit{ctx},\, \mathsf{op}(\mathit{id}, \mathit{deps},
    \mathsf{cursor}(\langle k_1, k_2, \dots, k_{n-1} \rangle,\, k_n), \mathit{mut}) \evalto
    \mathit{ctx}[\, k_1 \,\mapsto\, \mathit{state} \,]$}
\end{prooftree}

\begin{prooftree}
\AxiomC{$k_1 \notin \mathrm{dom}(\mathit{ctx})$}
\AxiomC{$k_1 = \mathsf{listT}(k_\mathit{new})$}
\AxiomC{$\begin{matrix}
    \{\mathsf{next}(\mathsf{head}) \mapsto \mathsf{tail}\}, \mathsf{op}(\mathit{id}, \mathit{deps}, \\
    \mathsf{cursor}(\langle k_2, \dots, k_{n-1} \rangle,\, k_n), \mathit{mut}) \evalto \mathit{state}
    \end{matrix} $}
\LeftLabel{\textsc{Def-List}}
\TrinaryInfC{$\mathit{ctx},\, \mathsf{op}(\mathit{id}, \mathit{deps},
    \mathsf{cursor}(\langle k_1, k_2, \dots, k_{n-1} \rangle,\, k_n), \mathit{mut}) \evalto
    \mathsf{ctx}[\, k_1 \,\mapsto\, \mathit{state} \,]$}
\end{prooftree}

\begin{prooftree}
\AxiomC{$\mathit{val} \not= \texttt{[]} \,\wedge\, \mathit{val} \not= \texttt{\string{\string}}$}
\AxiomC{$\mathsf{regT}(k) \in \mathrm{dom}(\mathit{ctx})$}
\AxiomC{$\begin{matrix}
    \mathit{concurrent} = \{ \mathit{id}_i \mapsto v_i \mid
    (\mathit{id}_i \mapsto v_i) \in \mathit{ctx}(\mathsf{regT}(k)) \\
    \wedge\, \mathit{id}_i \notin \mathit{deps} \}
    \end{matrix} $}
\LeftLabel{$\textsc{Assign}_1$}
\TrinaryInfC{$\mathit{ctx},\, \mathsf{op}(\mathit{id}, \mathit{deps},
    \mathsf{cursor}(\langle\rangle,\, k), \mathsf{assign}(\mathit{val})) \evalto
    \mathsf{ctx}[\, \mathsf{regT}(k) \,\mapsto\,
    \mathit{concurrent}[\, \mathit{id} \,\mapsto\, \mathit{val} \,]\,]$}
\end{prooftree}

\begin{prooftree}
\AxiomC{$\mathit{val} \not= \texttt{[]} \,\wedge\, \mathit{val} \not= \texttt{\string{\string}}$}
\AxiomC{$\mathsf{regT}(k) \notin \mathrm{dom}(\mathit{ctx})$}
\LeftLabel{$\textsc{Assign}_2$}
\BinaryInfC{$\mathit{ctx},\, \mathsf{op}(\mathit{id}, \mathit{deps},
    \mathsf{cursor}(\langle\rangle,\, k), \mathsf{assign}(\mathit{val})) \evalto
    \mathsf{ctx}[\, \mathsf{regT}(k) \,\mapsto\, \{\, \mathit{id} \,\mapsto\, \mathit{val} \,\}\,]$}
\end{prooftree}

\begin{prooftree}
\AxiomC{$\mathit{ctx}(\mathsf{next}(\mathit{prev})) = \mathit{next}$}
\AxiomC{$\mathit{next} < \mathit{id} \,\vee\, \mathit{next} = \mathsf{tail}$}
\AxiomC{$\mathit{ctx},\, \mathsf{op}(\mathit{id}, \mathit{deps},
    \mathsf{cursor}(\langle\rangle,\, \mathit{id}), \mathsf{assign}(\mathit{val})) \evalto \mathit{ctx}'$}
\LeftLabel{$\textsc{Insert}_1$}
\TrinaryInfC{$\mathit{ctx},\, \mathsf{op}(\mathit{id}, \mathit{deps},
    \mathsf{cursor}(\langle\rangle,\, \mathit{prev}), \mathsf{insert}(\mathit{val})) \evalto
    \mathit{ctx}'[\,\mathsf{next}(\mathit{prev}) \,\mapsto\, \mathit{id},\;
    \mathsf{next}(\mathit{id}) \,\mapsto\, \mathit{next}\,]$}
\end{prooftree}

% TODO need to update the next two rules

\begin{prooftree}
\AxiomC{$A_p(\mathit{prev}) = \mathsf{listEl}(v_\mathit{prev}, \mathit{next})$}
\AxiomC{$\mathit{id} < \mathit{next}$}
\AxiomC{$A_p,\, \mathsf{insertOp}(\mathit{id}, \mathit{next}, v) \evalto A_p'$}
\LeftLabel{$\textsc{Apply-Ins}_2$}
\TrinaryInfC{$A_p,\, \mathsf{insertOp}(\mathit{id}, \mathit{prev}, v) \evalto A_p'$}
\end{prooftree}

\begin{prooftree}
\AxiomC{$A_p(\mathit{target}) = \mathsf{listEl}(v, \mathit{next})$}
\AxiomC{$A_p,\, \mathsf{ts}(\mathit{id}) \evalto A_p'$}
\LeftLabel{\textsc{Apply-Del}}
\BinaryInfC{$A_p,\, \mathsf{deleteOp}(\mathit{id}, \mathit{target}) \evalto
    A_p'[\,\mathit{target} \,\mapsto\, \mathsf{listEl}(\bot, \mathit{next})\,]$}
\end{prooftree}
\caption{Rules for evaluating operations and modifying document state}\label{fig:operation-rules}
\end{figure*}



\ifproofdraft

When the operation is applied to the list state $A$, it produces a modified list state $A'$ as follows:
\begin{align*}
A' &= \mathrm{apply}(A, \mathsf{insert}(\mathit{id}, \mathit{prev}, v)) \\ &=
\begin{cases}
\quad \mathrm{apply}(A, \mathsf{insert}(\mathit{id}, n, v)) \\
    \qquad\quad\text{if }\; A(\mathit{prev}) = (\placeholder, n) \;\wedge\; \mathit{id} < n \\
\quad A[\,\mathit{prev} \mapsto (v_p, \mathit{id}),\; \mathit{id} \mapsto (v, n)\,] \\
    \qquad\quad\text{if }\; A(\mathit{prev}) = (v_p, n) \;\wedge\; n < \mathit{id}
\end{cases}
\end{align*}
where $<$ is the total order on Lamport timestamps, with the additional requirement that $\mathsf{tail} < (c, p)$ and $\mathsf{head} < (c, p)$ for any Lamport timestamp $(c, p)$.

Applying an $\mathsf{insert}$ operation is like inserting an element into a linked list, except that the function first skips over list elements with an ID greater than the ID of the new element being inserted. This has the effect of deterministically ordering concurrent insertions made at the same position of the list. This property is proved formally below.

The operation $\mathsf{delete}(\mathit{id})$ is an instruction to delete the element with ID $\mathit{id}$ from the list. The operation has the following semantics:
\begin{align*}
A' &= \mathrm{apply}(A, \mathsf{delete}(\mathit{id})) \\ &=
A[\,\mathit{id} \mapsto (\bot, n)\,]
\quad\text{if }\; A(\mathit{id}) = (\placeholder, n)
\end{align*}

The user view of a list A iterates over the list elements from $\mathsf{head}$ to $\mathsf{tail}$, skipping any deleted elements (tombstones):
\begin{align*}
\mathrm{view}(A) &= \mathrm{itr}(A, \mathsf{head}) \\
\mathrm{itr}(A, \mathit{id}) &= \begin{cases}
    [v] \mathbin{::} \mathrm{itr}(A, n) & \text{if }\; A(\mathit{id}) = (v, n) \;\wedge\; v \neq \bot \\
                     \mathrm{itr}(A, n) & \text{if }\; A(\mathit{id}) = (v, n) \;\wedge\; v = \bot \\
    [\,] & \text{if }\; \mathit{id} = \mathsf{tail}
\end{cases}
\end{align*}

\subsection{Convergence}

We now prove the convergence property, also known as \emph{eventual consistency}, for this data structure.

\textbf{Theorem.} If any two peers process the same set of $\mathsf{insert}$ and $\mathsf{delete}$ operations, in any causally consistent order (but not necessarily the same order), then their final states are identical.

\textbf{Proof.} Each peer processes operations in some sequential order, and the state of a peer is modified only by operations. Therefore, we can derive the state of peer $p$ from the history of operations $H_p=o_1 \dots o_n$ applied at that peer, processed in the order of the history:
$$ A_{H_p} = \mathrm{apply}(\mathrm{apply}(\dots \mathrm{apply}(A_\emptyset, o_1), \dots, o_{n-1}), o_n). $$

We can prove the theorem by induction over the length of the history $n$.

\emph{Base case:} A history of $n=0$ operations describes an empty document. An empty document is deterministically defined, so any two peers that have not executed any operations are by definition in the same state.

\emph{Induction step:} Given causal histories $H_1$ and $H_2$ of length $n$, such that $H_1=o_1 \dots o_n$ and $H_2$ is a permutation of $H_1$, and such that applying $H_1$ results in the same state as applying $H_2$, we can construct new histories of length $n+1$ by inserting an operation $o_{n+1}$ at any causally ready position in either $H_1$ or $H_2$. We must then show that for all of the histories constructed this way, applying the operations results in the same final state.

We show the induction step case by case, considering each type of operation $o_{n+1}$ that may be inserted.

\subsubsection{Commutativity of deletion}

First consider the case that $o_{n+1}=\mathsf{delete}(\mathit{id})$. A list element can only be deleted if it exists in the list, so by causality, there must exist an operation $o_i=\mathsf{insert}(\mathit{id}, \placeholder, \placeholder)$ at a prior point in the history. Moreover, the Lamport timestamp used as $\mathit{id}$ is unique, so there must be exactly one such $\mathsf{insert}$ operation, and there cannot be any $\mathsf{insert}$ operation for the same $\mathit{id}$ at a point after $o_{n+1}$ in the history.

Note that applying $\mathsf{delete}(\mathit{id})$ modifies only the mapping for $A(\mathit{id})$, and otherwise leaves the state $A$ unchanged, so it is trivially commutative with respect to any operation that does not modify or depend on $A(\mathit{id})$.

The first kind of operation that can modify the same $A(\mathit{id})$ is a $\mathsf{delete}$ operation for the same list element $\mathit{id}$. As the rule for applying this operation is idempotent, the presence of any other $\mathsf{delete}(\mathit{id})$ operations before or after $o_{n+1}$ in the history does not have any effect on the final state.

We showed that an $\mathsf{insert}(\mathit{id}, \placeholder, \placeholder)$ operation can only occur prior to $\mathsf{delete}(\mathit{id})$. Thus, the only other kind of operation that can depend on or modify $A(\mathit{id})$ is an $\mathsf{insert}(\placeholder, \mathit{id}, \placeholder)$ operation, or a recursive call to the $\mathrm{apply}$ function that uses such an operation internally. Note that when applying such an $\mathsf{insert}$ operation, the queries for $A(\mathit{prev})$ ignore the $\mathit{value}$ element of the tuple and leave it unchanged, and only examine and modify the $\mathit{next}$ element of the tuple. By contrast, applying $\mathsf{delete}(\mathit{id})$ modifies only the $\mathit{value}$ element of $A(\mathit{id})$, and leaves the $\mathit{next}$ element unchanged.

Thus, if $o_{n+1}=\mathsf{delete}(\mathit{id})$ is inserted at any causally ready point in a history, the final state does not depend on the insertion point, because the inserted operation does not interact with any prior or following operations.

\subsubsection{Commutativity of insertion}

Now consider the case that $o_{n+1}=\mathsf{insert}(\mathit{id}, \mathit{prev}, v)$. The Lamport timestamp $\mathit{id}$ is freshly generated by the peer on which the edit was performed. Let $p$ by the peer on which $o_{n+1}$ was generated, let $A_p$ be the state of $p$ immediately before $o_{n+1}$ was applied, and let $A_p'$ be the state of $p$ immediately after $o_{n+1}$ was applied.

By the definition of Lamport timestamps, $\mathit{id}$ is the greatest timestamp in $A_p'$, and greater than any timestamp occurring in $A_p$. Therefore, in the state update $A_p' = \mathrm{apply}(A_p, \mathsf{insert}(\mathit{id}, \mathit{prev}, v)),$ the non-recursive case ($n < \mathit{id}$) must apply. Therefore we have:
\begin{align*}
A_p' =\; & A_p[\,\mathit{prev} \mapsto (v_\mathit{prev}, \mathit{id}),\; \mathit{id} \mapsto (v, \mathit{next})\,] \\
\text{where}\quad & A_p(\mathit{prev}) = (v_\mathit{prev}, \mathit{next}) \\
& \mathit{prev} < \mathit{id} \quad\text{and}\quad \mathit{next} < \mathit{id}.
\end{align*}

The rules that modify a peer's state never remove an ID (i.e.\ the domain of the partial function $A$ monotonically grows as operations are applied). Thus, when $o_{n+1}$ is applied at another peer $q$ at any causally ready time in its history, $A_q(\mathit{prev})$ and $A_q(\mathit{next})$ must be defined in the state $A_q$, or $\mathit{next} = \mathsf{tail}$.

Note that the linked list structure is only modified by applying $\mathsf{insert}$ operations ($\mathsf{delete}$ operations modify values, but not the order of items in the list). Causal ordering requires $o_{n+1}$ to be applied after any operation that happened before, and before any operation that happened causally later, so we need only consider all possible orderings of $o_{n+1}$ with respect to other concurrent $\mathsf{insert}$ operations.

Moreover, the list structure is only modified by inserting a new element between two existing, adjacent elements. When $o_{n+1}$ is applied at peer $q$, the elements identified by $\mathit{prev}$ and $\mathit{next}$ are not necessarily adjacent in $A_q$, but we know that they must still be in the same order. According to the observation above, we also know that any other ID $\mathit{id}'$ inserted between those elements must have $\mathit{prev} < \mathit{id}'$ and $\mathit{next} < \mathit{id}'$.

By the induction hypothesis, let $H_1$ and $H_2$ be causal histories of operations $o_1 \dots o_n$ which result in the same final state when applied. When $o_{n+1}$ is inserted into either of these histories, causality demands that list elements $\mathit{prev}$ and $\mathit{next}$ were already inserted at some point in the history prior to $o_{n+1}$. When $o_{n+1}$ is applied, the $\mathrm{apply}(\placeholder, \mathsf{insert}(\mathit{id}, \mathit{prev}, v))$ rule starts iterating at $\mathit{prev}$ and skips over any elements with an ID greater than $\mathit{id}$. Since $\mathit{next} < \mathit{id}$, the recursion never continues beyond $\mathit{next}$. Thus, the $\mathrm{apply}$ rule is guaranteed to insert $\mathit{id}$ at some position between $\mathit{prev}$ and $\mathit{next}$.

Therefore, we need only consider insertions that are concurrent to $o_{n+1}$ and that insert at some position between $\mathit{prev}$ and $\mathit{next}$. Let $H_c$ be a history of operations, derived from either $H_1$ or $H_2$ by selecting only $\mathsf{insert}$ operations that are concurrent with $o_{n+1}$ and that insert at a position within the interval from $\mathit{prev}$ to $\mathit{next}$. By the induction hypothesis, all possible sub-histories $H_c$ result in the same final order of list elements within that interval. The operations in $H_c$ do not affect (and are not affected by) the list order outside of that interval, so we can consider $H_c$ in isolation.

Let $[\mathit{id}_1, \dots, \mathit{id}_k]$ be the Lamport timestamps of the list elements in the interval between $\mathit{prev}$ and $\mathit{next}$, in the list order unambiguously defined by $H_c$. We need to show that regardless at what position in $H_c$ we insert the operation $o_{n+1}$, the outcome is the same sequence of list elements $[\mathit{id}_1, \dots, \mathit{id}_m, \mathit{id}, \mathit{id}_{m+1}, \dots, \mathit{id}_k]$, where $\mathit{id}$ is the Lamport timestamp of $o_{n+1}$.

Since $\mathsf{insert}$ operations only add a new element between two existing, adjacent elements, it is sufficient to show that regardless where $o_{n+1}$ is inserted to $H_c$, the set of list elements $\{\mathit{id}_1, \dots, \mathit{id}_m\}$ that appear before $\mathit{id}$ in the final order is the same.

Note that the $\mathrm{apply}$ rule ensures that after $\mathit{id}$ is inserted, all list elements between $\mathit{prev}$ and the insertion point have an ID greater than $\mathit{id}$, and the list element following the insertion point has an ID less than $\mathit{id}$. Further note that by the definition of Lamport timestamps, any operation that causally depends on an operation $o_i$ must have a greater ID than $o_i$.

Let the set $T_\mathit{left} = \{\mathit{id}_i \mid \mathsf{insert}(\mathit{id}_i, \mathit{prev}, \placeholder) \in H_c\}$ contain the IDs of all operations in $H_c$ that use $\mathit{prev}$ as their reference position. Any history consisting only of these operations, in any order, results in the same final list order, namely in order of descending Lamport timestamp (TODO prove this in a lemma, it might not be obvious). Since Lamport timestamps are totally ordered, this list order is deterministic and unique. $\mathit{id}$ has a deterministic position within this order.

Any operation $o_i$ in $H_c$ that uses a different list element (not $\mathit{prev}$) as its reference position appears at some position after its reference element in the final list order, and has a Lamport timestamp greater than its reference element. If the reference element appears after $o_{n+1}$ in the final order, $o_i$ also appears after $o_{n+1}$, regardless of the order in which operations are applied.

If the reference element appears before $o_{n+1}$ in the final order, that means the timestamp of the reference element is greater than $\mathit{id}$, and so the timestamp of $o_i$ is also greater than $\mathit{id}$. If $o_{n+1}$ appears after $o_i$ in the history, then the apply rule for $o_{n+1}$ skips over both $o_i$ and the reference element. If $o_i$ appears after $o_{n+1}$ in the history, the apply rule for $o_i$ does not skip over the list element for $o_{n+1}$, again because the timestamp for $o_i$ is greater than $\mathit{id}$. Either way, if the reference element appears before $o_{n+1}$ in the final order, so does $o_i$, regardless of the order in which operations are applied.

TODO that last bit of the argument is a bit hand-wavy, need to make it more precise. But I think the general approach is ok. That completes the induction step, and thus the proof.

\fi % proofdraft

{\footnotesize
\bibliographystyle{plainnat}
\bibliography{references}{}}
\end{document}
